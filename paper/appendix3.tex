\section{Appendix: Code Listing}
\vspace{1em}
\definecolor{bg}{rgb}{0.95,0.95,0.95}

\begin{myjulia}
# Define precision
const Float = Float64

# Define state space datatype
mutable struct Q
    h :: Float
    hu :: Float
end

# Define spatial domain datatype
QS = Array{Q,1}
\end{myjulia}

\begin{myjulia}
# Make Q a vector space

import Base.+, Base.-, Base.*

function +(lhs::Q, rhs::Q)
    Q(lhs.h + rhs.h, lhs.hu + rhs.hu)
end

function -(lhs::Q, rhs::Q)
    Q(lhs.h - rhs.h, lhs.hu - rhs.hu)
end

function *(a::Real, q::Q)
    Q(a*q.h, a*q.hu)
end
\end{myjulia}

\begin{myjulia}
"Shallow-water point-wise flux"
function f(q :: Q)
    Q(q.hu, q.hu^2/q.h + 0.5*G*q.h^2)
end

"Shallow-water point-wise max wave speed"
function wavespeed(q :: Q)
    u = q.hu / q.h
    c = sqrt(G * q.h)
    max(abs(u-c), abs(u+c))
end

\end{myjulia}

\begin{myjulia}

"Finite volume skeleton"
function run(stoptime::Real, ncells::Int, choose_dt,
             compute_fluxes!, apply_bcs!, apply_ics)

    qs = apply_ics(ncells)
    Fl = QS(ncells+2)
    Fr = QS(ncells+2)

    currenttime = 0
    timesteps = 0
    dx = Float(1000.0/ncells)
    
    while currenttime < stoptime
        apply_bcs!(qs)
        dt = choose_dt(qs, dx)
        compute_fluxes!(qs, Fl, Fr, ncells, dx, dt)
        for x = 2:ncells+1
            qs[x] -= dt/dx * (Fr[x] + Fl[x-1])
        end
        currenttime += dt
        timesteps += 1
    end
    return (qs, currenttime, timesteps)
end
\end{myjulia}
\begin{myjulia}

"Lax-Friedrichs kernel"
function lxf!(qs::QS, Fl::QS, Fr::QS, 
              ncells::Int, dx::Float, dt::Float)
    
    a = dx/dt
    for x = 2:ncells+2
        ql,qr = qs[x-1], qs[x]
        fl,fr = f(ql), f(qr)        

        Fr[x-1] = 0.5*((fr-fl) - a*(qr - ql))
        Fl[x-1] = 0.5*((fr-fl) + a*(qr - ql))
    end
end

\end{myjulia}
\newpage
\begin{myjulia}

"Local Lax-Friedrichs kernel"
function llxf!(qs::QS, Fl::QS, Fr::QS, 
               ncells::Int, dx::Float, dt::Float)
    
    for x = 2:ncells+2
        ql,qr = qs[x-1], qs[x]
        fl,fr = f(ql), f(qr)

        a = max(wavespeed(ql), wavespeed(qr))

        Fr[x-1] = 0.5*((fr-fl) - a*(qr-ql))
        Fl[x-1] = 0.5*((fr-fl) + a*(qr-ql))
    end
end
\end{myjulia}
\begin{myjulia}
# Timestepping strategies: (QS, dx) -> dt

function make_const_dt(dt::Float)
    return (qs, dx) -> dt
end

function cfl_dt(qs::QS, dx::Real, mu::Real)
    lambda = map(wavespeed,qs) |> maximum
    return Float(dx / lambda * mu)
end

function make_cfl_dt(mu::Real)
    return (qs, dx) -> cfl_dt(qs, dx, mu)
end

function make_lagging_dt(dt0::Real, mu::Real)
    dt = dt0
    function lagging_dt(qs, dx)
        result = dt
        dt = cfl_dt(qs, dx, mu)
        return result
    end
    return lagging_dt
end
\end{myjulia}

\begin{myjulia}
# Initial conditions: ncells -> QS

function gaussian_ics(ncells::Int)
    qs = QS(ncells+2)
    mid = ncells / 2.0 + 1
    for x = 1:ncells+2
        pos_rel = (mid-x) / (ncells/10.0)
        qs[x] = Q(exp(-pos_rel^2) + 1, 0.0)
    end
    return qs
end

function breakingdam_ics(ncells::Int)
    [x < ncells/2 ? Q(2,0) : Q(1,0) for x in 1:ncells+2]
end
\end{myjulia}

\begin{myjulia}
# Boundary conditions: QS -> nothing 

function outflow_bcs!(qs::QS)
    qs[1] = qs[2]
    qs[end] = qs[end-1]
end

function periodic_bcs!(qs::QS)
    qs[1] = qs[end-1]
    qs[end] = qs[2]
end

function reflecting_bcs!(qs::QS)
    qs[1].h = qs[2].h
    qs[1].hu = -qs[2].hu
    qs[end].h = qs[end-1].h
    qs[end].hu = -qs[end-1].hu
end
\end{myjulia}

\begin{myjulia}

# Sample experiment run
qs,t,ts = run(50, 1000, make_cfl_dt(0.5), 
              lxf!, outflow_bcs!, breakingdam_ics)

\end{myjulia}
